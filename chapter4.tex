\chapter{Completely Regular Semigroups}


\section{Completely Regular Semigroups and Clifford Semigroups}


\begin{definition}
  A completely regular semigroup (CRS) contains data $(S, m(-, -), (-)')$, in which $S$ is a set; $m: S^2 \to S$ is a mapping satisfying the condition of associative that for all $a, b, c $ in $S$, $m(a, m(b, c)) = m(m(a. b), c)$, namely, $(S, m)$ forms a semigroup; and $(-)'$ is a mapping satisfying
  \begin{itemize}
    \item $\forall a \in S ((a')' = a)$;
    \item $\forall a \in S (aa' = a'a)$;
    \item $\forall a \in S (aa'a = a)$.
  \end{itemize}
\end{definition}



\begin{definition}
  A Clifford semigroup is a CRS $(S, \cdot, (-)')$ satisfies for all $, y \in S$
  \[
    (xx')(yy') = (yy')(xx').
  \]
\end{definition}


\begin{definition}
  A center of a semigroup $S$ is defined as 
  \[
    Z(S) = \{ a \in S: \forall x\in S (xa = ax) \}.
  \]
\end{definition}







From the definitions above, we immediately obtain some natures of CRS.
\begin{proposition}
  \label{prop:crs-crsandcli-props1} Let $S$ be a CRS, then
  \begin{enumerate}
    \item\label{crs-crsandcli-props1-1} for any $a \in S$, $a'aa' = a'$, thus $a' \in \inv(a)$;
    \item\label{crs-crsandcli-props1-2} for any $a \in S$, all of $a', aa', a'a$ belong to $\calH(a)$;
    \item\label{crs-crsandcli-props1-3} if $e \in S$ is an idempotent, $e = e' = ee' = e'e$.
  \end{enumerate}
\end{proposition}
\begin{proof}
  \ref{crs-crsandcli-props1-1} Notice that $a'aa' = a'(a')'a' = a'$.

  \ref{crs-crsandcli-props1-2} Observe that $a'$ is the inverse of $a$, that $aa' \in \calR(a) \cap \calL(a')$ and $a'a \in \calR(a') \cap \calL(a)$ and that $aa' = a'a$. Thus, $\calR(a) \cap \calL(a') = \calR(a') \cap \calL(a)$, which implies $\calR(a) \cap \calR(a') \neq \emptyset$ and $\calL(a) \cap \calL(a') \neq \emptyset$, so we may deduce that $a \calR a'$ and $a \calL a'$.

  \ref{crs-crsandcli-props1-3} Notice that $e = ee'e = eee' = ee', e' = e'ee' = e'e = ee' = e$.
\end{proof}

% From the definition above, we arrive at some brief natures. Suppose $S$ is a c.r.s, then
% \begin{nabls}{CRP1}
%   \nabitem{crgrp-crgrp-crp1-1} for $a \in S$, $a'aa' = a'$, thus $a' \in \inv(a)$;
%   \begin{proof}
%     \[
%       a'aa' = a'(a')'a' = a'.
%     \]
%   \end{proof}
%   \nabitem{crgrp-crgrp-crp1-2} for all $a \in S$, all $a', aa'. a'a$ belong to $\calH(a)$;
%   \begin{proof}
%     Observe that $a'$ is the inverse of $a$, that $aa' \in \calR(a) \cap \calL(a')$ and $a'a \in \calR(a') \cap \calL(a)$ by \ref{green-rg-reb-aa}, and that $aa' = a'a$. Thus, $\calR(a) \cap \calL(a') = \calR(a') \cap \calL(a)$, which implies $\calR(a) \cap \calR(a') \neq \emptyset$ and $\calL(a) \cap \calL(a') \neq \emptyset$, so we may deduce that $a \calR a'$ and $a \calL a'$.
%   \end{proof}
%   \nabitem{crgrp-crgrp-crp1-3} if $e \in S$ is an idempotent, $e = e' = ee' = e'e$.
%   \begin{proof}
%     \[
%       e = ee'e = eee' = ee', e' = e'ee' = e'e = ee' = e.
%     \]
%   \end{proof}
% \end{nabls}





\begin{proposition}
  \label{prop:crs-crsandcli-props2}
  Let $S$ be a semigroup, the following propositions are equivalent.
  \begin{enumerate}
    \item\label{crs-crsandcli-props2-1} $S$ is CR;
    \item\label{crs-crsandcli-props2-2} every element is $S$ lies in a subgroup of $S$;
    \item\label{crs-crsandcli-props2-3} every $\calH$-class is a group.
  \end{enumerate}
\end{proposition}
\begin{proof}
  \ref{crs-crsandcli-props2-1} $\Rightarrow$ \ref{crs-crsandcli-props2-2} Given a CRS $(S, \cdot, (-)')$, for any $a \in S$, let $e = a'a = aa'$. Then by \ref{crs-crsandcli-props1-2}, $a$ belongs to a group $\calH(e)$.

  \ref{crs-crsandcli-props2-2} $\Rightarrow$ \ref{crs-crsandcli-props2-3} Given a semigroup $(S, \cdot)$ that each element $a$ lies in a subgroup $G_a$ of $S$, define the mapping $(-)'$ that sends $a$ to its inverse in $G_a$. Then, from $aa' = a'a =: e_a$ and $e_a a = a e_a = a$, it follows that $a \calH e_a$, where $\calH(e_a)$ is a group, for $e_a$ is an idempotent.

  \ref{crs-crsandcli-props2-3} $\Rightarrow$ \ref{crs-crsandcli-props2-1} Suppose $(S, \cdot)$ is a semigroup that every $\calH$-calss within it is a group. It's easy to verify that the mapping, $(-)'$, sending $a$ to its inverse $a' \in \calH(a)$, together with the presupposed structure on $S$, forms a CRS.
\end{proof}








\begin{proposition}
  \label{prop:crs-crsandcli-props3}
  Suppose $S$ is a semigroup, the following statements are equivalent:
  \begin{enumerate}
    \item\label{crs-crsandcli-props3-1} $S$ is CS.
    \item\label{crs-crsandcli-props3-2} $S$ is CR, and for all $x, y \in S$, $xx' = (xyx)(xyx)'$;
    \item\label{crs-crsandcli-props3-3} $S$ is CR and simple.
  \end{enumerate}
\end{proposition}
\begin{proposition}
  \ref{crs-crsandcli-props3-1} $\Rightarrow$ \ref{crs-crsandcli-props3-2} By \ref{zssandzs-css-props1-7} of Proposition \ref{prop:zssandzs-css-props1}, every $\calH$-class of $S$ is a group, so we let $(-)'$ be the mapping that sending $a$ to its inverse within $\calH(a)$. Let $x, y \in S$, by \ref{zssandzs-css-props1-6} of Proposition \ref{prop:zssandzs-css-props1}, we obtain that $xy \in \calR(x) \cap \calL(y)$ and that $xyx \in \calR(xy) \cap \calL(x)$, in which $\calR(xy) = \calR(x)$. Thus, $xyx \in \calH(x)$. Hence, $xx' = (xyx)(xyx)'$.

  \ref{crs-crsandcli-props3-2} $\Rightarrow$ \ref{crs-crsandcli-props3-3} Suppose $(S, \cdot, (-)')$ is a CRS and $a, b \in S$. Then 
  \[
    a = aa'a = a\cdot b \cdot a(aba)'a,
  \]
  and so $\calJ(a) \leq \calJ(b)$. By interchanging the role of $a$ and $b$, we may equally show that $\calJ(b) \leq \calJ(a)$. Thus, $\calJ(a) = \calJ(b)$ for any $a , b$ in $S$, and so $\calJ = S^2$, which implies $S$ is simple.

  \ref{crs-crsandcli-props3-3} $\Rightarrow$ \ref{crs-crsandcli-props3-1} This is a direct conclusion of Proposition \ref{prop:zssandzs-css-props4}.
\end{proposition}





Regarding the last derivation above, we have an even stronger conclusion.

\begin{proposition}
  \label{prop:crs-crsandcli-prop4}
  Suppose $(S, \cdot, (-)')$ is simple and CR, then every idempotent of $S$ is primitive.
\end{proposition}
\begin{proof}
  Let $e, f$ be two idempotents of $S$, and $e \leq f$, namely, $ef = fe = e$. Since $S$ is simple, there exists $z, t \in S$ such that $e = zft$. Let $x = ezf$ and $y = fte$, then, it can be verified that
  \[
    xfy = e, ex = xf = e, fy = ye = y.
  \] 
  On the other hand, $S$ is CR, so $x$ lies in a group $\calH$-class, in which there exists an identity $e_x$ satisfies $e_x x = x e_x = x$ and $x'x = xx' = e_x$. Observe that
  \[
    f = e_x f = e_x e f = e_x xfyf = xfy = e,
  \]
  so we may conclude that every idempotent of $S$ is primitive.
\end{proof}







We Stipulate that a partially ordered set $(I, \leq)$ can be made into a category, where for any elements $i, j$ of $I$,
\[
  i \to j \Leftrightarrow j \leq i.
\]
In particular, a semilattice can also be made into a category in the same manner.

Given a category $\calC$ and a small category $I$, one can define a functor $\Delta_I$, which maps each $S$ to the functor $\Delta_I(S)$ that sends $i$ to $S$ and $i\to j$ to $\idd_S$, and maps each morphism $S \to T$ in $\calC$ to the morphism of functors $\Delta_I(S) \to \Delta_I(T)$ that send every $i$ to the certain morphism $S \to T$.

A colimit $\varinjlim \alpha$ of functor $\alpha: I \to \calC$, is the initial object within the comma category $(j_{\alpha}, \Delta_I)$, that is, for any pair $(L, f)$, in which $L$ is an object of $\calC$ and $f: \alpha \to \Delta_I(L)$ is a morphism of functors, there exists unique morphism $\phi: \varinjlim \alpha \to L$ that makes the following diagram commutes.
\[
  \begin{tikzcd}
    \alpha
      \arrow[r]
      \arrow[rd, "f"']
    &\Delta_I(\varinjlim \alpha)
      \arrow[d, dashed, "\exists !"', "\Delta_I(\phi)"]
    \\
    &\Delta_I(L)
  \end{tikzcd}
\] 







\begin{proposition}
  \label{prop:crs-crsandcli-smltsofsmg}
  Let $(I, \leq, \land)$ be a semilattice, $\alpha: I \to \cate{Smg}$ be a functor and $S$ be the set $\bigsqcup_{i\in I}\alpha(i)$. Define a multiplication on $S$ as
  \[
    (x, i)(y, j) := (\alpha_{i \to i \land j}(x) \alpha_{j \to i\land j}(y), i \land j);
  \]
  and a family of functions $\iota(i): \alpha(i) \to S$ that maps $x$ to $(x, i)$ for every $i \in I$, which are in fact the natural inclusions. Then, it can be verified that:
  \begin{enumerate}
    \item\label{crs-crsandcli-smltsofsmg-1} the multiplication defined above is well-defined;
    \item\label{crs-crsandcli-smltsofsmg-2} each $\iota(i): \alpha(i) \to S$ is a morphism of semigroups;
    \item\label{crs-crsandcli-smltsofsmg-3} $S$, together with the morphism of functors $\iota = [i \mapsto \iota(i)]: \alpha \to \Delta_I(S)$, forms the colimit of $\alpha$.
  \end{enumerate} 
\end{proposition}







\begin{proposition}
  Let $\alpha: I \to \cate{Smg}$ be a functor, where $I$ is a filtered category (a partial ordered set is of course filtered). Let $S$ be the set $\bigsqcup_{i \in I}\alpha(i)$. The relation, $(x, i) \sim (y, j) \Leftrightarrow \exists k \in \Ob(I)$ such that $i \rightarrow k \leftarrow j$ and $\alpha_{i \to k}(x) = \alpha_{j \to k}(y)$, is an equivalence relation. We denote by $[x, i]$ the equivalence class of $(x, i)$. Define a multiplication on $S/\sim$ as 
  \[
    [x, i][y, j] := [\alpha_{i \to k}(x) \alpha_{j \to k}(y), k],
  \]
  and a family of functions $\iota(i): \alpha(i) \to S/\sim$ that maps $x$ to $[x, i]$. Then, it can be verified that:
  \begin{enumerate}
    \item the multiplication defined above is well defined, thus $S/\sim$ forms a semigroup;
    \item each $\iota(i)$ is a morphism of semigroups;
    \item $S/\sim$, together with the morphism of functors $\iota = (\iota(i))_i: \alpha \to \Delta_I(S/\sim)$, forms the colimit of $\alpha$.
  \end{enumerate}
\end{proposition}






% \begin{definition}
%   We say that $(S, \cdot)$ is a semilattice of CSSs (GRPs), if
%   \begin{itemize}
%     \item $(I, \leq, \land)$ is a semilattice;
%     \item $\alpha: I \to \Ob(\cate{Css})$ is a mapping;
%     \item as a set, $S = \bigsqcup \alpha(i)$ and each $(\alpha(i), \cdot)$ is a subsemigroup of $(S, \cdot)$;
%     \item for any $i, j$ in $I$, $\alpha(i) \cdot \alpha(j) \subset \alpha(i\land j)$.
%   \end{itemize}

%   Furthermore, we say that $(S, \cdot)$ is a strong semilattice of CSSs, if
%   \begin{itemize}
%     \item $(I, \leq, \land)$ is a semilattice;
%     \item $\alpha: I \to \cate{Css}$ is a functor;
%     \item $S = \bigsqcup_{i \in I} \alpha(i)$;
%     \item for any $(x, i), (y, j) \in S$
%     \[
%       (x, i)\cdot (y, j) = (\alpha_{i \to i\land j}(x)\alpha_{j \to i\land j}(y), i\land j),
%     \]
%     thus still it holds that $((x, i)(y, j))(z, k) = (x, i)((y, j)(z, k))$, that $\alpha(i) \cdot \alpha(j) \subset \alpha(i\land j)$, and that each $\alpha(i) \xhookrightarrow{\enspace} S$ is a subsemigroup.
%     \end{itemize}
% \end{definition}







% \begin{proposition}
%   \label{prop:crgrp-crgrp-jc}
%   Suppose $(S, \cdot, (-)')$ is CR., then
%   \begin{enumerate}
%     \item\label{crgrp-crgrp-jc-1} $\forall a \in S (\calJ(a) = \calJ(a^2))$;
%     \item\label{crgrp-crgrp-jc-2} $\forall a, b \in S (\calJ(ab) = \calJ(ba))$;
%     \item\label{crgrp-crgrp-jc-3} $\calJ$ is a congruence, thus, $(S/\calJ, \ast)$ is a semigroup, where $\calJ(a) \ast \calJ(b) := \calJ(ab)$.
%     \item\label{crgrp-crgrp-jc-4} $\calJ(a) \leq \calJ(b) \Leftrightarrow \calJ(a) = \calJ(a)\ast \calJ(b)$, thus $(\calJ, \leq, \land)$ consists a semilattice, where $\calJ(a)\land \calJ(b) := \inf \{ \calJ(a), \calJ(b) \} = \calJ(a)\ast \calJ(b)$;
%     \item\label{crgrp-crgrp-jc-5} $S$ is a semilattice of CSSs.
%   \end{enumerate}
% \end{proposition}

% \begin{proof}
%   \ref{crgrp-crgrp-jc-1} Notice that $a'a^2 = a, aa = a^2$ and that $a^2a' = a, aa=a^2$, clearly $a \calH a^2$, thus $a \calJ a^2$.

%   \ref{crgrp-crgrp-jc-2} For any $a, b \in S$, 
%   \[
%     \calJ(ab) = \calJ(abab) \leq \calJ(ba).
%   \]
%   Similarly we have $\calJ(ba) \leq \calJ(ab)$, so $\calJ(ab) = \calJ(ba)$.

%   \ref{crgrp-crgrp-jc-3} If $a \calJ b$, then $b = xay, a = ubv$ for some $x,y,u,v$ in $S^1$. If $c \in S$, then
%   \[
%     \calJ(ca) = \calJ(cubv) \leq \calJ(cub) = \calJ(ubc) \leq \calJ(bc) = \calJ(cb);
%   \]
%   similarly one can establish that $\calJ(cb) \leq \calJ(ca)$. By virtue of previous result we obtain $\calJ(ac) = \calJ(bc)$, and so $\calJ$ is a congruence.

%   \ref{crgrp-crgrp-jc-4} Hitherto we have established that $(S/\calJ, \ast)$ is a commutative semigroup, in which every element is idempotent. If $\calJ(a) = \calJ(ab)$, then $(a) = (ab) \subset (b)$, and it follows that $\calJ(a) \leq \calJ(b)$. Conversely, suppose $\calJ(a) \leq \calJ(b)$, there exists $x, y \in S^1$ such that $xby=a$, and so
%   \[
%     \calJ(a) = \calJ(a^2) = \calJ(axby) \leq \calJ(axb) = \calJ(bax) \leq \calJ(ba) = \calJ(ab).
%   \]
%   By $(ab) \subset (a)$ we also have $\calJ(ab) \leq \calJ(a)$. Thus, $\calJ(ab) = \calJ(a)$.

%   \ref{crgrp-crgrp-jc-5} We shall show that every $\calJ(p) \in S/\calJ$ is a CSS. Here we denote by $A$ the set $\calJ(p)$ for convenience. The property $\calJ(p)\cdot \calJ(p) \subset \calJ(p^2) = \calJ(p)$ give that $A$ is a semigroup. We assert that $\calJ^A = A^2$, thus it is simple, and since it is also CR., it must be c.s. For any $a, b \in A$, there exists $x,y,u,v \in S$ such that $xay = b, ubv = a$. There exists idempotents $e, f$ so that $a \in \calH(e) \subset A$ and $b \in \calH(f) \subset A$ on the grounds of $S$ is CR. Hence
%   \[
%     fxayf = fbf = b, eubvf = eae = a;
%   \]
%   it's clear that $\calJ(fx) \geq \calJ(fxayf) = \calJ(b) = A$, and similarly $yf, eu, ve$ are all in $A$. The equations above mean that $a \calJ^A b$, thus $\calJ^A = A^2$.

%   To show that $A$ is CR., observe that any $a \in A$ lies in a group $\calH(a) \subset \calJ(p)$, thus by Proposition \ref{prop:crgrp-crgrp-eqp}, it is a c.r.s.

%   To complete this proof, we define the mapping $\idd: S/\calJ \to S/\calJ: \alpha \mapsto \alpha$, view $(S/\calJ, \leq, \land)$ as a semilattice, $S$ as the disjoint union $\bigsqcup_{\alpha \in S/\calJ} \idd (\alpha)$. Since for any $\calJ$-calsses $\alpha$ and $\beta$, $\alpha \cdot \beta \subset \alpha \land \beta$, $S$ becomes a semilattice of CSSs.
% \end{proof}







% \begin{proposition}
%   A strong semilattice of CRSs is also a CRS.
% \end{proposition}
% \begin{proof}
%   The multiplication defined on $S$ is as follows: for any $(x, i), (y, j) \in S$
%   \[
%     (x, i)\cdot (y, j) = (\alpha_{i \to i\land j}(x)\alpha_{j \to i\land j}(y), i\land j),
%   \]
%   still it holds that $((x, i)(y, j))(z, k) = (x, i)((y, j)(z, k))$ and that $\alpha(i) \cdot \alpha(j) \subset \alpha(i\land j)$.
%   Define the mapping $(a, i) \mapsto (a', i)$, where $a'$ is the result of $(-)'$ belongs to CRS $\alpha(i)$. Thus, 
%   \[
%     (a, i)(a', i) = (\alpha_{i \to i\land i}(a)\alpha_{i \to i \land i}(a'), i\land i) = (aa', i);
%   \]
%   the remaining verifications are all straightforward.
% \end{proof}










% \begin{theorem}
%   \label{theo:crgrp-crgrp-clfeqs}
%   Suppose $S$ is a semigroup, then the following statements are equivalent:
%   \begin{enumerate}
%     \item\label{crgrp-crgrp-clfeqs-1} $S$ is a Clifford semigroup;
%     \item\label{crgrp-crgrp-clfeqs-2} $S$ is a semilattice of groups;
%     \item\label{crgrp-crgrp-clfeqs-3} $S$ is a strong semilattice of groups;
%     \item\label{crgrp-crgrp-clfeqs-4} $S$ is regular, and $\idm S \subset Z(S)$;
%     \item\label{crgrp-crgrp-clfeqs-5} $S$ is regular, and $\calD^S \cap (\idm S)^2  = \Delta_{\idm S}$.
%   \end{enumerate}
% \end{theorem}

% \begin{proof}
%   \ref{crgrp-crgrp-clfeqs-1} $\Rightarrow$ \ref{crgrp-crgrp-clfeqs-2} Let $S$ be a Clifford semigroup, by \ref{crgrp-crgrp-jc-5} of Proposition \ref{prop:crgrp-crgrp-jc}, we first obtain that $S$ is a semilattice of CSSs. Consider a particular CSS $S_i$. Since every idempotent $e = ee'$, the idempotents in $S_i$ are commutative. By Proposition \ref{prop:crgrp-crgrp-idmispri}, every idempotent in $S_i$ is primitive, then for any $e, f \in \idm S_i$,
%   \[
%     e(ef) = ef(e) = ef, f(ef) = (ef)f = ef,
%   \]
%   which means that $ef \leq e$ and $ef \leq f$. Since  both $e, f$ are primitive, $e = ef = f$, that is, $S_i$ contains only one idempotent. According to the property that any $\calH$-class of the CSS $S_i$ is a group, which naturally contains an idempotent, so we may deduce that $S_i$ has only one $\calH$-class, implying $S_i$ is a group.

%   \ref{crgrp-crgrp-clfeqs-2} $\Rightarrow$ \ref{crgrp-crgrp-clfeqs-3} This statement essentially says that the mapping $\alpha: I \to \Ob(\cate{Grp})$ within a semilattice of groups can be used to construct a functor. As will be evident in the proof below, the well-behaved properties of groups make this upgrade straightforward.  We denote the element in $\alpha(i)$ by $x_i$. Observe that $\left(\bigsqcup_{i \in I}\alpha(i), \cdot\right)$ as a semilattice of groups, satisfies $\alpha(i) \cdot \alpha(j) \subset \alpha(i\land j)$ and every $(\alpha(i), \cdot)$ is the subsemigroup of $\left(\bigsqcup_{i \in I}\alpha(i), \cdot\right)$.
  
%   Suppose $i \to j$, define the mapping $\alpha_{i \to j}: \alpha(i) \to \alpha(j)$ as $x_i \mapsto e_j \cdot x_i$, where $e_j$ is the identity of group $\alpha(j)$, and the remaining verifications proceed in these parts:
%   \begin{itemize}
%     \item for any $i \to j$, $\alpha_{i \to j}$ is a morphism;
%     \item for any $i\in \Ob(I)$, $\alpha_{i \to i} = \idd_{\alpha(i)}$;
%     \item for any $i \to j \to k$ within $I$, $\alpha_{j \to k}\circ \alpha_{i \to j} = \alpha_{i \to k}$;
%     \item for any $i, j \in \Ob(I)$, the multiplication that presupposed on $S$ coincides with the multiplication defined as follows,
%     \[
%       x_i \bullet y_j = \alpha_{i \to i\land j}(x) \cdot \alpha_{j \to i\land j}(y).
%     \]
%   \end{itemize}
%   See \cite[Theorem 4.2.1]{John-FundSmg} for details.

%   \ref{crgrp-crgrp-clfeqs-3} $\Rightarrow$ \ref{crgrp-crgrp-clfeqs-4} Defining $x'_i := x_i^{-1}$ which lies in the group $\alpha(i)$, makes $S$ a regular semigroup. Its idempotents are the identity elements $e_i$ of the group $\alpha(i)$. For any $x_j \in \alpha(j) \subset S$,
%   \[
%     e_i\cdot x_j = \alpha_{i \to i\land j} (e_i) \alpha_{j \to i\land j} (x_j) =  e_{i\land j} \alpha_{j \to i\land j} (x_j) = \alpha_{j \to i\land j} (x_j) = x_j \cdot e_i,
%   \]
%   thus, $\idm S \subset Z(S)$.

%   \ref{crgrp-crgrp-clfeqs-4} $\Rightarrow$ \ref{crgrp-crgrp-clfeqs-5}  Suppose $e \calD f$, by (...) there exists $a$ and its inverse $a'$ such that $a'a = f, aa'=e$. So we have
%   \[
%     e = e^2 = aa'aa' = afa' = faa' = a'aaa' = a'ae = a'ea = a'aa'a = f^2 = f.
%   \]

%   \ref{crgrp-crgrp-clfeqs-5} $\Rightarrow$ \ref{crgrp-crgrp-clfeqs-1} If a $\calD$-calss contains a single idempotent $e$, then $\calD(e)$ is regular, and for any $a \in \calD(e)$ there exists $x$ such that  $axa = a$. It follows that $ax$ is an idempotent belongs to $\calR(a)$, then so it equals to $e$; similarly $xa = e$. From $ea = a$ and $ae = a$, we obtain $a \calH e$, thus $\calD(e) = \calH(e)$. 
% \end{proof}

 