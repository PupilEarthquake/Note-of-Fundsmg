\chapter{Completely Regular Semigroups}


\section{Completely Regular Semigroups and Clifford Semigroups}


\begin{definition}
  A completely regular semigroup (CRS) contains data $(S, m(-, -), (-)')$, in which $S$ is a set; $m: S^2 \to S$ is a mapping satisfying the condition of associative that for all $a, b, c $ in $S$, $m(a, m(b, c)) = m(m(a. b), c)$, namely, $(S, m)$ forms a semigroup; and $(-)'$ is a mapping satisfying
  \begin{itemize}
    \item $\forall a \in S ((a')' = a)$;
    \item $\forall a \in S (aa' = a'a)$;
    \item $\forall a \in S (aa'a = a)$.
  \end{itemize}
\end{definition}



\begin{definition}
  A Clifford semigroup is a CRS $(S, \cdot, (-)')$ satisfies for all $, y \in S$
  \[
    (xx')(yy') = (yy')(xx').
  \]
\end{definition}


\begin{definition}
  A center of a semigroup $S$ is defined as 
  \[
    Z(S) = \{ a \in S: \forall x\in S (xa = ax) \}.
  \]
\end{definition}







From the definitions above, we immediately obtain some natures of CRS.
\begin{proposition}
  \label{prop:crs-crsandcli-props1} Let $S$ be a CRS, then
  \begin{enumerate}
    \item\label{crs-crsandcli-props1-1} for any $a \in S$, $a'aa' = a'$, thus $a' \in \inv(a)$;
    \item\label{crs-crsandcli-props1-2} for any $a \in S$, all of $a', aa', a'a$ belong to $\calH(a)$;
    \item\label{crs-crsandcli-props1-3} if $e \in S$ is an idempotent, $e = e' = ee' = e'e$.
  \end{enumerate}
\end{proposition}
\begin{proof}
  \ref{crs-crsandcli-props1-1} Notice that $a'aa' = a'(a')'a' = a'$.

  \ref{crs-crsandcli-props1-2} Observe that $a'$ is the inverse of $a$, that $aa' \in \calR(a) \cap \calL(a')$ and $a'a \in \calR(a') \cap \calL(a)$ and that $aa' = a'a$. Thus, $\calR(a) \cap \calL(a') = \calR(a') \cap \calL(a)$, which implies $\calR(a) \cap \calR(a') \neq \emptyset$ and $\calL(a) \cap \calL(a') \neq \emptyset$, so we may deduce that $a \calR a'$ and $a \calL a'$.

  \ref{crs-crsandcli-props1-3} Notice that $e = ee'e = eee' = ee', e' = e'ee' = e'e = ee' = e$.
\end{proof}

% From the definition above, we arrive at some brief natures. Suppose $S$ is a c.r.s, then
% \begin{nabls}{CRP1}
%   \nabitem{crgrp-crgrp-crp1-1} for $a \in S$, $a'aa' = a'$, thus $a' \in \inv(a)$;
%   \begin{proof}
%     \[
%       a'aa' = a'(a')'a' = a'.
%     \]
%   \end{proof}
%   \nabitem{crgrp-crgrp-crp1-2} for all $a \in S$, all $a', aa'. a'a$ belong to $\calH(a)$;
%   \begin{proof}
%     Observe that $a'$ is the inverse of $a$, that $aa' \in \calR(a) \cap \calL(a')$ and $a'a \in \calR(a') \cap \calL(a)$ by \ref{green-rg-reb-aa}, and that $aa' = a'a$. Thus, $\calR(a) \cap \calL(a') = \calR(a') \cap \calL(a)$, which implies $\calR(a) \cap \calR(a') \neq \emptyset$ and $\calL(a) \cap \calL(a') \neq \emptyset$, so we may deduce that $a \calR a'$ and $a \calL a'$.
%   \end{proof}
%   \nabitem{crgrp-crgrp-crp1-3} if $e \in S$ is an idempotent, $e = e' = ee' = e'e$.
%   \begin{proof}
%     \[
%       e = ee'e = eee' = ee', e' = e'ee' = e'e = ee' = e.
%     \]
%   \end{proof}
% \end{nabls}





\begin{proposition}
  \label{prop:crs-crsandcli-props2}
  Let $S$ be a semigroup, the following propositions are equivalent.
  \begin{enumerate}
    \item\label{crs-crsandcli-props2-1} $S$ is CR;
    \item\label{crs-crsandcli-props2-2} every element is $S$ lies in a subgroup of $S$;
    \item\label{crs-crsandcli-props2-3} every $\calH$-class is a group.
  \end{enumerate}
\end{proposition}
\begin{proof}
  \ref{crs-crsandcli-props2-1} $\Rightarrow$ \ref{crs-crsandcli-props2-2} Given a CRS $(S, \cdot, (-)')$, for any $a \in S$, let $e = a'a = aa'$. Then by \ref{crs-crsandcli-props1-2}, $a$ belongs to a group $\calH(e)$.

  \ref{crs-crsandcli-props2-2} $\Rightarrow$ \ref{crs-crsandcli-props2-3} Given a semigroup $(S, \cdot)$ that each element $a$ lies in a subgroup $G_a$ of $S$, define the mapping $(-)'$ that sends $a$ to its inverse in $G_a$. Then, from $aa' = a'a =: e_a$ and $e_a a = a e_a = a$, it follows that $a \calH e_a$, where $\calH(e_a)$ is a group, for $e_a$ is an idempotent.

  \ref{crs-crsandcli-props2-3} $\Rightarrow$ \ref{crs-crsandcli-props2-1} Suppose $(S, \cdot)$ is a semigroup that every $\calH$-calss within it is a group. It's easy to verify that the mapping, $(-)'$, sending $a$ to its inverse $a' \in \calH(a)$, together with the presupposed structure on $S$, forms a CRS.
\end{proof}








\begin{proposition}
  \label{prop:crs-crsandcli-props3}
  Suppose $S$ is a semigroup, the following statements are equivalent:
  \begin{enumerate}
    \item\label{crs-crsandcli-props3-1} $S$ is CS.
    \item\label{crs-crsandcli-props3-2} $S$ is CR, and for all $x, y \in S$, $xx' = (xyx)(xyx)'$;
    \item\label{crs-crsandcli-props3-3} $S$ is CR and simple.
  \end{enumerate}
\end{proposition}
\begin{proposition}
  \ref{crs-crsandcli-props3-1} $\Rightarrow$ \ref{crs-crsandcli-props3-2} By \ref{zssandzs-css-props1-7} of Proposition \ref{prop:zssandzs-css-props1}, every $\calH$-class of $S$ is a group, so we let $(-)'$ be the mapping that sending $a$ to its inverse within $\calH(a)$. Let $x, y \in S$, by \ref{zssandzs-css-props1-6} of Proposition \ref{prop:zssandzs-css-props1}, we obtain that $xy \in \calR(x) \cap \calL(y)$ and that $xyx \in \calR(xy) \cap \calL(x)$, in which $\calR(xy) = \calR(x)$. Thus, $xyx \in \calH(x)$. Hence, $xx' = (xyx)(xyx)'$.

  \ref{crs-crsandcli-props3-2} $\Rightarrow$ \ref{crs-crsandcli-props3-3} Suppose $(S, \cdot, (-)')$ is a CRS and $a, b \in S$. Then 
  \[
    a = aa'a = a\cdot b \cdot a(aba)'a,
  \]
  and so $\calJ(a) \leq \calJ(b)$. By interchanging the role of $a$ and $b$, we may equally show that $\calJ(b) \leq \calJ(a)$. Thus, $\calJ(a) = \calJ(b)$ for any $a , b$ in $S$, and so $\calJ = S^2$, which implies $S$ is simple.

  \ref{crs-crsandcli-props3-3} $\Rightarrow$ \ref{crs-crsandcli-props3-1} This is a direct conclusion of Proposition \ref{prop:zssandzs-css-props4}.
\end{proposition}





Regarding the last derivation above, we have an even stronger conclusion.

\begin{proposition}
  \label{prop:crs-crsandcli-prop4}
  Suppose $(S, \cdot, (-)')$ is simple and CR, then every idempotent of $S$ is primitive.
\end{proposition}
\begin{proof}
  Let $e, f$ be two idempotents of $S$, and $e \leq f$, namely, $ef = fe = e$. Since $S$ is simple, there exists $z, t \in S$ such that $e = zft$. Let $x = ezf$ and $y = fte$, then, it can be verified that
  \[
    xfy = e, ex = xf = e, fy = ye = y.
  \] 
  On the other hand, $S$ is CR, so $x$ lies in a group $\calH$-class, in which there exists an identity $e_x$ satisfies $e_x x = x e_x = x$ and $x'x = xx' = e_x$. Observe that
  \[
    f = e_x f = e_x e f = e_x xfyf = xfy = e,
  \]
  so we may conclude that every idempotent of $S$ is primitive.
\end{proof}







We Stipulate that a partially ordered set $(I, \leq)$ can be made into a category, where for any elements $i, j$ of $I$,
\[
  i \to j \Leftrightarrow j \leq i.
\]
In particular, a semilattice can also be made into a category in the same manner.

% Given a category $\calC$ and a small category $I$, and $\alpha\in \calC^I$ is an object of functor category. one can define a functor $\Delta_I$, which maps each $S$ to the functor $\Delta_I(S): I \to \calC$ that sends $i$ to $S$ and $i\to j$ to $\idd_S$, and maps each morphism $S \to T$ in $\calC$ to the morphism of functors $\Delta_I(S) \to \Delta_I(T)$ that send every $i$ to the certain morphism $S \to T$.

% A colimit $\varinjlim \alpha$ of functor $\alpha: I \to \calC$, is the initial object within the comma category $(j_{\alpha}, \Delta_I)$, that is, for any pair $(L, f)$, in which $L$ is an object of $\calC$ and $f: \alpha \to \Delta_I(L)$ is a morphism of functors, there exists unique morphism $\phi: \varinjlim \alpha \to L$ that makes the following diagram commutes.
% \[
%   \begin{tikzcd}
%     \alpha
%       \arrow[r]
%       \arrow[rd, "f"']
%     &\Delta_I(\varinjlim \alpha)
%       \arrow[d, dashed, "\exists !"', "\Delta_I(\phi)"]
%     \\
%     &\Delta_I(L)
%   \end{tikzcd}
% \] 
% The definition of limits is similar, it can be found in any category textbook, and we omit it here.






\begin{definition}
  \label{def:crs-crsandcli-smltsofsmg}
  Let $(I, \leq, \land)$ be a semilattice, $\alpha: I \to \cate{Smg}$ be a functor and $S$ be the set $\bigsqcup_{i\in I}\alpha(i)$. Define a multiplication on $S$ as
  \[
    (x, i)\ast(y, j) := (\alpha_{i \to i \land j}(x) \alpha_{j \to i\land j}(y), i \land j);
  \]
  and a family of functions $\iota(i): \alpha(i) \to S$ that maps $x$ to $(x, i)$ for every $i \in I$, which are in fact the natural inclusions. Then, it can be verified that:
  \begin{enumerate}
    \item\label{crs-crsandcli-smltsofsmg-1} each $\iota(i): \alpha(i) \to S$ is a morphism of semigroups, but the series of these morphisms, $\iota = [i\mapsto \iota(i)]$, may not consist a natural transformation between $\alpha \to \Delta_I(S)$;
    \item\label{crs-crsandcli-smltsofsmg-2} the set $S$, together with the multiplication defined above, forms a semigroup;
    \item\label{crs-crsandcli-smltsofsmg-3} for any $i, j$ in $I$, $\alpha(i) \ast \alpha(j) \subset \alpha(i\land j)$;
    \item\label{crs-crsandcli-smltsofsmg-4} for any semigroup $L$ and the morphism $f: \alpha \to \Delta_I(L)$, there exists unique morphism $\phi: S \to L$ which is given by $(x, i) \mapsto f(i)(x)$ such that for any $i \in I$ the following diagram commuts.
    \[
      \begin{tikzcd}
        \alpha(i)
          \arrow[r, hook, "\iota(i)"]
          \arrow[rd, "f(i)"']
        &S
          \arrow[d, dashed, "\exists !"', "\phi"]
        \\
        &L
      \end{tikzcd}
    \]
  \end{enumerate}
  Hence, we call the semigroup $(S, \ast)$ the \textit{strong semilattice of semigroups}.
\end{definition}







\begin{definition}
  Let $(I, \leq, \land)$ be a semilattice, $(S, \ast)$ be a semigroup, $\alpha: I \to \Ob(\cate{Smg})$ be a mapping and $(\iota(i): \alpha(i) \hookrightarrow S)_{i \in I}$ be a series of semigroup monomorphisms. We say that $S$ is the \textit{lattice of semigroups} if
  \begin{enumerate}
    \item $\alpha(i) \ast \alpha(j) \subset \alpha(i\land j)$;
    \item under the perspective of set, $S = \bigsqcup_{i \in I} \alpha(i)$.
  \end{enumerate}
\end{definition}







The above two definitions, the strong semilattice of semigroups and the semilattice of semigroups, can generalize to the cases of CRSs, CSSs and groups.







\begin{proposition}
  Suppose $I$ is a semilattice and $\alpha: I \to \cate{Css}$ is a functor, then still is a CSS the semigroup $(S, \ast)$ constructed by Definition \ref{def:crs-crsandcli-smltsofsmg}.
\end{proposition}
\begin{proof}
  For any element $(x, i)$ in $S$, let $(x, i)'$ be $(x', i)$ where $x'$ is the inverse of $x$ within $\calH(x)$ by the property of CSSs that every $\calH$-calss is a group. It can be verified that $(-)'$ satisfies the conditions of \ref{crs-crsandcli-props3-2} of Proposition \ref{prop:crs-crsandcli-props3}, and so it is a CSS.
\end{proof}







\begin{example}
  Let $\alpha: I \to \cate{Smg}$ be a functor, where $I$ is a filtered category (a partial ordered set is of course filtered). Let $S$ be the set $\bigsqcup_{i \in I}\alpha(i)$. The relation, $(x, i) \sim (y, j) \Leftrightarrow \exists k \in \Ob(I)$ such that $i \rightarrow k \leftarrow j$ and $\alpha_{i \to k}(x) = \alpha_{j \to k}(y)$, is an equivalence relation. We denote by $[x, i]$ the equivalence class of $(x, i)$. Define a multiplication on $S/\sim$ as 
  \[
    [x, i][y, j] := [\alpha_{i \to k}(x) \alpha_{j \to k}(y), k],
  \]
  and a family of functions $\iota(i): \alpha(i) \to S/\sim$ that maps $x$ to $[x, i]$. Then, it can be verified that:
  \begin{enumerate}
    \item the multiplication defined above is well defined, thus $S/\sim$ forms a semigroup;
    \item each $\iota(i)$ is a morphism of semigroups, and the series of these morphisms $\iota = [i \mapsto \iota(i)]$ is a natural transformation between functors $\alpha$ and $\Delta_I(S)$;
    \item $S/\sim$, together with the morphism of functors $\iota$, forms the colimit of $\alpha$.
  \end{enumerate}
\end{example}




















% \begin{definition}
%   We say that $(S, \cdot)$ is a semilattice of CSSs (GRPs), if
%   \begin{itemize}
%     \item $(I, \leq, \land)$ is a semilattice;
%     \item $\alpha: I \to \Ob(\cate{Css})$ is a mapping;
%     \item as a set, $S = \bigsqcup \alpha(i)$ and each $(\alpha(i), \cdot)$ is a subsemigroup of $(S, \cdot)$;
%     \item for any $i, j$ in $I$, $\alpha(i) \cdot \alpha(j) \subset \alpha(i\land j)$.
%   \end{itemize}

%   Furthermore, we say that $(S, \cdot)$ is a strong semilattice of CSSs, if
%   \begin{itemize}
%     \item $(I, \leq, \land)$ is a semilattice;
%     \item $\alpha: I \to \cate{Css}$ is a functor;
%     \item $S = \bigsqcup_{i \in I} \alpha(i)$;
%     \item for any $(x, i), (y, j) \in S$
%     \[
%       (x, i)\cdot (y, j) = (\alpha_{i \to i\land j}(x)\alpha_{j \to i\land j}(y), i\land j),
%     \]
%     thus still it holds that $((x, i)(y, j))(z, k) = (x, i)((y, j)(z, k))$, that $\alpha(i) \cdot \alpha(j) \subset \alpha(i\land j)$, and that each $\alpha(i) \xhookrightarrow{\enspace} S$ is a subsemigroup.
%     \end{itemize}
% \end{definition}







\begin{proposition}
  \label{prop:crs-crsandcli-jc}
  Let $(S, \cdot, (-)')$ be a CRS, then
  \begin{enumerate}
    \item\label{crs-crsandcli-jc-1} $\forall a \in S (\calJ(a) = \calJ(a^2))$;
    \item\label{crs-crsandcli-jc-2} $\forall a, b \in S (\calJ(ab) = \calJ(ba))$;
    \item\label{crs-crsandcli-jc-3} $\calJ$ is a congruence, thus $(S/\calJ, \ast)$ is a semigroup, where $\calJ(a) \ast \calJ(b) := \calJ(ab)$;
    \item\label{crs-crsandcli-jc-4} $\calJ(a) \leq \calJ(b) \Leftrightarrow \calJ(a) = \calJ(a)\ast \calJ(b)$, thus $(\calJ, \leq, \land)$ consists a semilattice, where $\calJ(a)\land \calJ(b) := \inf \{ \calJ(a), \calJ(b) \} = \calJ(a)\ast \calJ(b)$;
    \item\label{crs-crsandcli-jc-5} $S$ is a semilattice of CSSs.
  \end{enumerate}
\end{proposition}

\begin{proof}
  \ref{crs-crsandcli-jc-1} Notice that $a'a^2 = a, aa = a^2$ and that $a^2a' = a, aa=a^2$, clearly $a \calH a^2$, thus $a \calJ a^2$.

  \ref{crs-crsandcli-jc-2} For any $a, b \in S$, 
  \[
    \calJ(ab) = \calJ(abab) \leq \calJ(ba).
  \]
  Similarly we have $\calJ(ba) \leq \calJ(ab)$, so $\calJ(ab) = \calJ(ba)$.

  \ref{crs-crsandcli-jc-3} If $a \calJ b$, then $b = xay, a = ubv$ for some $x,y,u,v$ in $S^1$. If $c \in S$, then
  \[
    \calJ(ca) = \calJ(cubv) \leq \calJ(cub) = \calJ(ubc) \leq \calJ(bc) = \calJ(cb);
  \]
  similarly one can establish that $\calJ(cb) \leq \calJ(ca)$. By virtue of previous result we obtain $\calJ(ac) = \calJ(bc)$, and so $\calJ$ is a congruence.

  \ref{crs-crsandcli-jc-4} Hitherto we have established that $(S/\calJ, \ast)$ is a commutative semigroup, in which every element is idempotent. If $\calJ(a) = \calJ(ab)$, then $(a) = (ab) \subset (b)$, and it follows that $\calJ(a) \leq \calJ(b)$. Conversely, suppose $\calJ(a) \leq \calJ(b)$, there exists $x, y \in S^1$ such that $xby=a$, and so
  \[
    \calJ(a) = \calJ(a^2) = \calJ(axby) \leq \calJ(axb) = \calJ(bax) \leq \calJ(ba) = \calJ(ab).
  \]
  By $(ab) \subset (a)$ we also have $\calJ(ab) \leq \calJ(a)$. Thus, $\calJ(ab) = \calJ(a)$.

  \ref{crs-crsandcli-jc-5} We shall show that every $\calJ(p) \in S/\calJ$ is a CSS. Here we denote by $A$ the set $\calJ(p)$ for convenience. The property $\calJ(p)\cdot \calJ(p) \subset \calJ(p^2) = \calJ(p)$ give that $A$ is a semigroup. We assert that $\calJ^A = A^2$, thus it is simple, and due to it is also CR, it must be CS. For any $a, b \in A$, there exists $x,y,u,v \in S$ such that $xay = b, ubv = a$. There exists idempotents $e, f$ such that $a \in \calH(e) \subset A$ and $b \in \calH(f) \subset A$ on the grounds of $S$ is CR. Hence
  \[
    fxayf = fbf = b, eubvf = eae = a;
  \]
  it's clear that $\calJ(fx) \geq \calJ(fxayf) = \calJ(b) = A$, and similarly all of $yf, eu, ve$ are in $A$. The equations above mean that $a \calJ^A b$, thus $\calJ^A = A^2$.

  To show $A$ is CR, observe that any $a \in A$ lies in a group $\calH(a) \subset \calJ(p)$.

  To complete this proof, define the mapping $\idd: S/\calJ \to S/\calJ: \alpha \mapsto \alpha$, where $(S/\calJ, \leq, \land)$ can be viewed as a semilattice. On the other hand $S$ is the disjoint union $\bigsqcup_{\alpha \in S/\calJ} \alpha$; and for any $\calJ$-calsses $\alpha$ and $\beta$, $\alpha \cdot \beta \subset \alpha \land \beta$; thus, $S$ becomes a semilattice of CSSs.
\end{proof}







% \begin{proposition}
%   A strong semilattice of CRSs is also a CRS.
% \end{proposition}
% \begin{proof}
%   The multiplication defined on $S$ is as follows: for any $(x, i), (y, j) \in S$
%   \[
%     (x, i)\cdot (y, j) = (\alpha_{i \to i\land j}(x)\alpha_{j \to i\land j}(y), i\land j),
%   \]
%   still it holds that $((x, i)(y, j))(z, k) = (x, i)((y, j)(z, k))$ and that $\alpha(i) \cdot \alpha(j) \subset \alpha(i\land j)$.
%   Define the mapping $(a, i) \mapsto (a', i)$, where $a'$ is the result of $(-)'$ belongs to CRS $\alpha(i)$. Thus, 
%   \[
%     (a, i)(a', i) = (\alpha_{i \to i\land i}(a)\alpha_{i \to i \land i}(a'), i\land i) = (aa', i);
%   \]
%   the remaining verifications are all straightforward.
% \end{proof}










\begin{theorem}
  \label{theo:crs-crsandcli-clis}
  Suppose $S$ is a semigroup, then the following statements are equivalent:
  \begin{enumerate}
    \item\label{crs-crsandcli-clis-1} $S$ is a Clifford semigroup;
    \item\label{crs-crsandcli-clis-2} $S$ is a semilattice of groups;
    \item\label{crs-crsandcli-clis-3} $S$ is a strong semilattice of groups;
    \item\label{crs-crsandcli-clis-4} $S$ is regular, and $\idm S \subset Z(S)$;
    \item\label{crs-crsandcli-clis-5} $S$ is regular, and $\calD^S \cap (\idm S)^2  = \Delta_{\idm S}$.
  \end{enumerate}
\end{theorem}

\begin{proof}
  \ref{crs-crsandcli-clis-1} $\Rightarrow$ \ref{crs-crsandcli-clis-2} Let $S$ be a Clifford semigroup, by \ref{crs-crsandcli-jc-5} of Proposition \ref{prop:crs-crsandcli-jc}, we first obtain that $S$ is a semilattice of CSSs. Consider a particular CSS $S_i$. Since every idempotent $e = ee'$, the idempotents in $S_i$ are commutative. By Proposition \ref{prop:crs-crsandcli-prop4}, every idempotent in $S_i$ is primitive, then for any $e, f \in \idm S_i$,
  \[
    e(ef) = ef(e) = ef, f(ef) = (ef)f = ef,
  \]
  which means $ef \leq e$ and $ef \leq f$. Since  both $e, f$ are primitive, $e = ef = f$, that is, $S_i$ contains only one idempotent. According to the property that any $\calH$-class of the CSS $S_i$ is a group, which naturally contains an idempotent, so we may deduce that $S_i$ is a group on account of $S_i$ has only one $\calH$-class.

  \ref{crs-crsandcli-clis-2} $\Rightarrow$ \ref{crs-crsandcli-clis-3} This statement essentially says that the mapping $\alpha: I \to \Ob(\cate{Grp})$ within a semilattice of groups can be used to construct a functor. As will be evident in the proof below, the well-behaved properties of groups make this upgrade straightforward.  We denote the element in $\alpha(i)$ by $x_i$. Observe that $\left(\bigsqcup_{i \in I}\alpha(i), \cdot\right)$ as a semilattice of groups, satisfies $\alpha(i) \cdot \alpha(j) \subset \alpha(i\land j)$ and every $(\alpha(i), \cdot)$ is the subsemigroup of $\left(\bigsqcup_{i \in I}\alpha(i), \cdot\right)$.
  
  Suppose $i \to j$, define the mapping $\alpha_{i \to j}: \alpha(i) \to \alpha(j)$ as $x_i \mapsto e_j \cdot x_i$, where $e_j$ is the identity of group $\alpha(j)$, and the remaining verifications proceed in these parts:
  \begin{itemize}
    \item for any $i \to j$, $\alpha_{i \to j}$ is a morphism;
    \item for any $i\in \Ob(I)$, $\alpha_{i \to i} = \idd_{\alpha(i)}$;
    \item for any $i \to j \to k$ within $I$, $\alpha_{j \to k}\circ \alpha_{i \to j} = \alpha_{i \to k}$;
    \item for any $i, j \in \Ob(I)$, the multiplication that presupposed on $S$ coincides with the multiplication defined as follows,
    \[
      x_i \ast y_j = \alpha_{i \to i\land j}(x) \cdot \alpha_{j \to i\land j}(y).
    \]
  \end{itemize}
  See \cite[Theorem 4.2.1]{John-FundSmg} for details.

  \ref{crs-crsandcli-clis-3} $\Rightarrow$ \ref{crs-crsandcli-clis-4} Defining $x'_i := x_i^{-1}$ which lies in the group $\alpha(i)$, makes $S$ a regular semigroup. Its idempotents are the identity elements $e_i$ of the group $\alpha(i)$. For any $x_j \in \alpha(j) \subset S$,
  \[
    e_i\cdot x_j = \alpha_{i \to i\land j} (e_i) \alpha_{j \to i\land j} (x_j) =  e_{i\land j} \alpha_{j \to i\land j} (x_j) = \alpha_{j \to i\land j} (x_j) = x_j \cdot e_i,
  \]
  thus, $\idm S \subset Z(S)$.

  \ref{crs-crsandcli-clis-4} $\Rightarrow$ \ref{crs-crsandcli-clis-5}  Suppose $e \calD f$, by \ref{green-rs-props2-4} of Proposition \ref{prop:green-rs-props2} there exists $a$ and its inverse $a'$ such that $a'a = f, aa'=e$. So we have
  \[
    e = e^2 = aa'aa' = afa' = faa' = a'aaa' = a'ae = a'ea = a'aa'a = f^2 = f.
  \]

  \ref{crs-crsandcli-clis-5} $\Rightarrow$ \ref{crs-crsandcli-clis-1} If a $\calD$-calss contains a single idempotent $e$, we assert that $\calD(e)$ is a group. By \ref{green-rs-props1-3} of Proposition \ref{prop:green-rs-props1}, $\calD(e)$ is regular, so for any $a \in \calD(e)$ there exists $x \in S^1$ such that  $axa = a$. It follows that $ax$ is an idempotent belongs to $\calR(a) \subset \calD(e)$, which must coincides with $e$ due to the condition that $e$ is the unique idempotent of $\calD(e)$; similarly $xa = e$. From $ea = a$ and $ae = a$, we obtain $a \calH e$, thus $\calD(e) = \calH(e)$. 

  On account of every element of $S$ lies in a group, it is a CRS, and so is a semilattice of CSSs $S_i$. We assert that every $S_i$ is a group. For every $x, y$ in $S_i$, by \ref{zssandzs-css-props1-6} of Proposition \ref{prop:zssandzs-css-props1}, $xy \in \calR(x) \cap \calL(y)$, and so $x \calD y$, thus $S_i \subset \calD(y)$, which implies $S_i$ is a group as we've established in \ref{crs-crsandcli-clis-1} $\Rightarrow$ \ref{crs-crsandcli-clis-2} on the condition of $S_i$ contains a single idempotent.

  From \ref{crs-crsandcli-clis-2} $\Rightarrow$ \ref{crs-crsandcli-clis-3} we now deduce that $S$ is a strong semilattice of groups. Thus, one can define the operation $(-)'$ on $S$ that sends $x_i$ to its inverse $x_i'$ within the group $S_i$, and it follows easily that for any $x_i, y_j \in S$
  \[
    x_i x_i'y_jy_j' = e_i e_j = e_{i\land j} = e_j e_i = y_jy_j'x_ix_i'.
  \]
\end{proof}








\section{Varieties}

\begin{definition}
  An $\Omega$-algebra contains data $(X, \Omega, \ar, \ot)$, where, $X$ is a set, set $\Omega$ is called operator domain, the arity $\ar$ is a function $\Omega \to \Z_{\geq 1}$ and $\ot$ is still a function belongs to $\prod_{\omega \in \Omega} \Hom(X^{\ar X}, X)$.
\end{definition}



Particularly, when $\Omega$ contains two elements with the arity of $2$ and $1$ respectively, we call the system $(2,1)$-algebra. A semigroup is a $(2)$-algebra; a group can be reckoned as a $(2,1,1)$-algebra.

In this section, if not specified, $\Omega$ is set to be a finite set.



\begin{definition}
  Given the data $(\Omega, \ar)$, the category of $\Omega$-algebras $\Omega\text{-}\cate{Aj}$ contains the following data:
  \begin{itemize}
    \item objects: all $\Omega$-algebras in which every $\Omega$-algebra as a set is a $\calU$-small set, where $\calU$ is a Grothendieck universe;
    \item $\Hom((X, \Omega, \ar, \ot_X), (Y, \Omega, \ar, \ot_Y))$: function $f:X \to Y$ satisfies for all $\omega \in \Omega$, 
    \[
      f(\ot_X \omega(x_1, \cdots, x_{\ar \omega})) = \ot_Y \omega (f(x_1), \cdots, f(x_{\ar \omega}));
    \]
    \item morphisms: the union of all $\Hom$ sets;
    \item composition: the composition of functions.
  \end{itemize}
\end{definition}






Let $I$ be a discrete $\calU$-small category, $\cate{Aj}$ be a category of $\Omega$-algebras and $\beta: I \to \cate{Aj}$ be a functor. The image of one certain $i \in I$ under $\beta$ is an $\Omega$-algebra denoted as $(X_i, \Omega, \ar, \ot_i)$. For every $\omega \in \Omega$, define the function $\ot \omega: (\prod_i X_i)^{\ar \omega} \to \prod_i X_i$ as follows
\[
  (\bm{x}_1, \cdots, \bm{x}_{\ar \omega}) \mapsto \left( \ot \omega (x_{1, i}, \cdots, x_{\ar \omega, i}) \right)_{i \in I},
\]
where $\bm{x}_k = (x_{k,i})_{i \in I}$. A family of projection $p_j:\prod_{i} X_i \to X_j$ is given by $(x_i)_i \mapsto x_j$. We may deduce that for any $\Omega$-algebra $L$ and the natural transformation $f: \Delta(L) \to \beta$, there exists unique morphism $\phi$ that makes the following diagram commutes, that is, $\cate{Aj}$ possesses products.
\[
  \begin{tikzcd}
    \Delta(\prod_i X_i)
      \arrow[r, "p"]
    &\beta
    \\
    \Delta(L)
      \arrow[ru, "f"']
      \arrow[u, dashed, "\exists !", "\phi"']
    &
  \end{tikzcd}
\]






\begin{definition}
  $(U, \Omega, \ar, \ot_U)$ is the subalgebra of $(X, \Omega, \ar, \ot_X)$ if
  \begin{itemize}
    \item $U \subset X$;
    \item for any $\omega \in \Omega$, $(\ot_X \omega) | U^{\ar \omega} = \ot_U \omega$.
  \end{itemize}
\end{definition}
Additionally, if the subset $U \subset X$ satisfies $\ot_X (U^{\ar \omega}) \subset U$ for all $\omega$, the former can be viewed as a subalgebra with a natural definition of $\ot_U$.






\begin{definition}
  Given the algebra $(X, \Omega, \ar, \ot_X)$, $(X^2, \Omega, \ar, \ot_{X^2})$ is its product algebra, $C \subset X^2$ is a congruence if
  \begin{itemize}
    \item $C \in \Eqv(X)$,
    \item $\ot_{X^2} \omega (C^{\ar \omega}) \subset C$.
  \end{itemize} 
\end{definition}
Simply say, $x_1 C y_1 ,\cdots, x_m C y_m \Rightarrow x_1\cdots x_m C y_1 \cdots y_m$. Furthermore, we may notice that the intersection of a set of congruences is also a congruence.






\begin{definition}
  Let $X$ be an algebra, $C$ is its congruence, the algebra $X/C$ is defined as a set $\{ C(x): x\in X\}$ with the operators $\ot_{X/C} \omega: (C(x_1), \cdots, C(x_{\ar \omega})) \mapsto C(\ot_{X}\omega (x_1, \cdots, x_{\ar \omega}))$.
\end{definition}
If $f: X \to Y$ is a homomorphism, the relation $\ker f := \{ (x_1, x_2) \in X^2: f(x_1) = f(x_2) \}$ is a congruence, for $f(x_1\cdots x_m) = f(x_1) \cdots f(x_m) = f(y_1)\cdots f(y_m) = f(y_1\cdots y_m)$. Based on this, we may build the theorem commonly appears in algebra systems: if $C$ is a congruence contained in $\ker f$, then there exists a unique morphism $g$ that makes the following diagram commutes.
\[
  \begin{tikzcd}
    X
      \arrow[r, two heads]
      \arrow[d, "f"']
    & X/C
      \arrow[ld, dashed, "\exists !"', "g"]
    \\
    Y
  \end{tikzcd}
\]





\begin{definition}
  For any set $X$, the \textit{free algebra} of $X$, denoted as $\bmrm{F}(X)$, is the initial object within the comma category $(j_X, U)$, in which $U: \cate{Aj} \to \cate{Set}$ is a forgetful functor. In other words, for any $\Omega$-algebra $L$ and the mapping $f: X \to L$, there exists a unique \textit{morphism} $\bar f$ that makes the following diagram commutes.
  \[
    \begin{tikzcd}
      X
        \arrow[r, hook, "\iota"]
        \arrow[rd, "f"']
      &U\bmrm{F}(X)
        \arrow[d, dashed, "\bar f"]
      \\
      &UL
    \end{tikzcd}
  \]
\end{definition}

If every $X$ has a free algebra, given a mapping $f: X \to Y$, by the definition above, there exist a unique morphism $\bmrm{F}(f)$ such that the diagram below commutes. This implies $\bmrm{F}$ can be made into a functor. In fact, $\begin{tikzcd}
  \cate{Set}
    \arrow[r, bend left=30, "\bmrm{F}"]
  &\cate{Aj}
    \arrow[l, bend left=30, "U"]
\end{tikzcd}$ is an adjunction pair.
\[
  \begin{tikzcd}
    X
      \arrow[r, hook]
      \arrow[d, "f"']
    &U\bmrm{F}(X)
      \arrow[d, dashed, "\bmrm{F}f"]
    \\
    Y
      \arrow[r, hook]
    &U\bmrm{F}(Y)
  \end{tikzcd}
\]











Free algebras exist, and a construction will be given here. Let $F_0$ be the set $X \sqcup \Omega \sqcup \{ (, ) \}$. Note that we are still examining this problem within the framework of the ZFC system, and both the elements of $\Omega$ and the parentheses ``('', ``)'' can be replaced by some set. Besides this, we denote by $\bmrm{On}$ the class of all ordinals and by $\bmrm{V}$ the class of all sets. 



\begin{theorem}[Transfinite Recursion]
  Let $G: \bmrm{V} \to \bmrm{V}$ be a function, $\theta \in \bmrm{On}$, then there exists unique function $a: \theta \to \bmrm{V}$ such that for any $\alpha < \theta$
  \[
    a(\alpha) = G(a|_{\alpha}).
  \]
\end{theorem}



Let $S$ be the free semigroup $\bmrm{F}_{\cate{Smg}}(X \sqcup \Omega \sqcup \{ (, ) \})$. For any element $\omega \in \Omega$, define the function $H_{\omega}: S^{\ar \omega} \to S$ as $(u_1, \cdots, u_{\ar \omega}) \mapsto \omega((u_1)(u_2)\cdots(u_{\ar \omega}))$. The set $\bmrm{F}$ is set to be the smallest set containing $F_0$ and closed under every $H_{\omega}$, that is,
\[
  \bmrm{F} := \bigcap \left\{ T \subset S: F_0 \subset T \land \left( \forall \omega \in \Omega, \forall \bm{u} \in T^{\ar \omega} (H_{\omega}(\bm{u}) \in T) \right) \right\}.
\]
Furthermore, from $F_0$, one can recursively define a mapping $\bmrm{On} \to \bmrm{V}$:
\begin{align*}
  & F_0 = F_0 \\
  & F_{\alpha + 1} := F_{\alpha} \cup \{ H_{\omega}(\bm{u}): \omega \in \Omega \land \bm{u} \in (F_{\alpha})^{\ar \omega} \}\\
  &\text{(}\gamma\text{ is a limit ordinal) } F_{\gamma} := \bigcup_{\beta < \gamma} F_{\beta}.
\end{align*}

\begin{lemma}
  $F_{\aleph_0 + 1} = F_{\aleph_0} = \bmrm{F}$.
\end{lemma}
\begin{proof}
  It can be verified that $F_{\aleph_0}$ satisfies the properties that it contains $F_0$ and closed under every $H_{\omega}$. By definition of $\bmrm{F}$, it follows that $\bmrm{F} \subset F_{\aleph_0}$. Conversely, since $F_0 \subset \bmrm{F}$, so $F_1 \subset \bmrm{F}$, and so every $F_{\alpha} \subset \bmrm{F}$, where $\alpha \leq \aleph_0$.

  Clearly $F_{\aleph_0} \subset F_{\aleph_0 + 1} $, and the proof of reverse containment is straightforward.
\end{proof}






We assert that $\bmrm{F}$, together with the mapping $H: \omega \mapsto H_{\omega}$, forms an $\Omega$-algebra, and even becomes the free $\omega$-algebra of set $X$. The embedding mapping $\iota: X \hookrightarrow \bmrm{F}$ is obvious, to let $\phi$ be a homomorphism and ensure the diagram
\begin{equation}
  \label{eq:crs-var-freeaj}
  \begin{tikzcd}
    X
      \arrow[r, hook, "\iota"]
      \arrow[rd, "f"']
    &\bmrm{F}
      \arrow[d, dashed, "\phi"]
    \\
    &L
  \end{tikzcd}
\end{equation}
commutes, we define $\phi$ recursively as follows. Let $\theta = \aleph_0 + 1$ and $G: \bmrm{V} \to \bmrm{V}$ be the function:
\[
  \begin{tikzcd}[row sep = small]
    \bmrm{V}
      \arrow[r, "G"]
    &\bmrm{V}
    \\
    \emptyset
      \arrow[r, mapsto]
    &\left[ \phi_0: F_0 \to L: x \mapsto f(x) \right]
    \\
    \left\{ f_{\alpha}: F_{\alpha} \to L \right\}_{\alpha \leq \lambda < \aleph_0}
      \arrow[r, mapsto]
    &f_{\lambda + 1}
    \\
    \left\{ f_{\alpha}: F_{\alpha} \to L \right\}_{\alpha < \aleph_0}
      \arrow[r, mapsto]
    &\bigcup_{\beta < \aleph_0} \Gamma(f_{\beta})
    \\
    \left\{ f_{\alpha}: F_{\alpha} \to L \right\}_{\alpha \leq \aleph_0}
      \arrow[r, mapsto]
    &f_{\aleph_0 + 1}
    \\
    \text{else}
      \arrow[r, mapsto]
    &\chi_0 \notin L ,
  \end{tikzcd}
\]
where $f_{\lambda+1}$ is defined as:
\[
  u \mapsto 
  \left\{
    \begin{aligned}
      &f_{\lambda}(u) & u \in F_{\lambda} \\
      &\ot \omega (f_{\lambda}(a_1), \cdots, f_{\lambda}(a_{\ar \omega})) & u = H_{\omega}(a_1,\cdots, a_{\ar \omega}) \land a_i \in F_{\lambda} \land u \notin F_{\lambda}.
    \end{aligned}
  \right.
\]
Thus, by the transfinite recursion, there exists unique $a: \theta \to \bmrm{V}$, such that sends
\begin{align*}
  &0 \to [\phi_0: F_0 \to L],\\
  &1 \to [\phi_1: F_1 \to L], \\
  &\vdots\\
  &\aleph_0 \to [\phi: F_{\aleph_0} \to L],
\end{align*}
in which $\phi|_{F_{\alpha}} = \phi_{\alpha}$ for any $\alpha < \aleph_0$. And so it can be verified that $\phi$ is a morphism.




To prove the uniqueness, suppose $\phi, \psi$ are morphisms that both make the diagram (\ref{eq:crs-var-freeaj}) commutes. Let 
\[
  C = \{ \alpha < \theta : \phi|_{F_{\alpha}} = \psi|_{F_{\alpha}}\},
\]
then it follows that
\begin{itemize}
  \item $0 \in C$:
  \item $\alpha \in C \Rightarrow \alpha + 1 \in C$;
  \item if $\gamma$ is a limit ordinal and for every $\alpha < \gamma$, $\alpha \in C$, then $\gamma \in C$.
\end{itemize}
By transfinite induction, we may conclude that $C = \theta$.





\begin{definition}
  The \textit{variety} $\calV$ is a subset of $\Ob(\cate{Aj})$, satisfies the properties:
  \begin{itemize}
    \item $S \in \calV $ and $T$ is the subalgebra of $S$, then $T \in \calV$;
    \item $S \in \calV$ and there exist an epimorphism $f: S \to T$, then $T \in \calV$;
    \item given a small set $I$, $S_i \in \calV$ for any $i \in I$, them $\prod_i S_i \in \calV$.  
  \end{itemize}
  Furthermore, a variety could be naturally made into a full subcategory of $\cate{Aj}$.
\end{definition}






\begin{definition}
  Given a set $X$, the $\calV$-\textit{free algebra} $\bmrm{F}_{\calV}(X)$ is an object of $\calV$ satisfies for any $L \in \Ob{\calV}$ and any mapping $f: X \to L$, there exist a unique morphism $\hat f$ such that the following diagram commutes:
  \[
    \begin{tikzcd}
      X
        \arrow[r, hook, "\iota"]
        \arrow[rd, "f"']
      &\bmrm{F}_{\calV}(X)
        \arrow[d, dashed, "\hat f"]
      \\
      &L
    \end{tikzcd}
  \]
\end{definition}






% Now we return to the $(2)$-algebra, alternatively, we focus only on a certain binary operation on an $\Omega$-algebra, which works as a ``multiplication'' that $(x, y) \mapsto xy$, for we need to make use of congruence again. Suppose $X$ is a $(2,1)$-algebra, recall that $R \subset X^2$ is a congruence if $R$ is an equivalence and $(x, y) \in R \land (u,v) \in R \Rightarrow (xu, yv) \in R$. Assume that the morphism $f: X \to Y$ satisfies $R \subset \ker f$, then there exists a unique morphism that makes the following diagram commutes.
% \[
%   \begin{tikzcd}
%     X
%       \arrow[d, "f"']
%       \arrow[r, two heads, "R"]
%     &X/R
%       \arrow[ld, dashed, "\exists !"]
%     \\
%     Y
%   \end{tikzcd}
% \]





To construct the $\bmrm{F}_{\calV}(X)$, and to facilitate the subsequent descriptions in this section, we define the following notations.
\begin{definition}
  Let $X$ be a set, then
  \begin{itemize}
    \item for any $S \in \Ob \calV$, $\fker(f,X,S):=\ker\left[\bmrm{F}(X) \xrightarrow{\bar{f}} S\right]$, where $\bar{f}$ is derived from the definition of free algebra;
    \item $\MS_{\calV}(X):= \{ (f, S) \in \Mor(\cate{Set}) \times \Ob(\calV): f \in \Hom_{\cate{Set}}(X, S) \}$
    \item $\rmc_{\calV}(X) := \bigcap_{(f, S) \in \MS_{\calV}(X)} \fker(f, X, S)$;
    \item for $u, v$ in $\bmrm{F}(X)$, $\ec_{\calV}(X, u, v):= \left\{ S \in \Ob(\calV): (u, v) \in \bigcap_{f \in \Hom_{\cate{Set}}(X, S)} \fker(f, X, S) \right\}$;
    \item given a binary relation $R \subset \bmrm{F}(X)^2$, the equational class $\ec_{\calV}(X, R):= \bigcap_{(u,v)\in R}\ec_{\calV}(X, u, v)$.
  \end{itemize}
\end{definition}






Now we may observe that for any $S \in \Ob \calV$ and mapping $f: X \to S$, there exist unique morphism $\hat{f}: \bmrm{F}(X) / c_{\calV}(X)$, due to the commutative diagram below.
\[
  \begin{tikzcd}
    X
      \arrow[r, hook]
      \arrow[rd, "f"']
    &\bmrm{F}(X)
      \arrow[r, two heads]
      \arrow[d, dashed, "\exists !"', "\bar{f}"]
    &\bmrm{F}(X) / c_{\calV}(X)
      \arrow[ld, dashed, "\exists !"', "\hat{f}"]
    \\
    &S
    &
  \end{tikzcd}
\]
And the issue lies in proving that $\bmrm{F}(X) / c_{\calV}(X) \in \Ob \calV$. 


Notice that every $\bmrm{F}(X) / \fker(f, X, S) \simeq \im \bar{f} \subset S$ is a subalgebra, and so is an element of $\calV$. Define a morphism $\Theta$
\[
  \begin{tikzcd}[row sep = small]
    \bmrm{F}(X)
      \arrow[r]
    & \prod_{(f, S) \in \MS_{\calV}(X)} \bmrm{F}(X) / \fker(f, X, S) \\
    u
      \arrow[r, mapsto]
    &\left[ (f, S) \mapsto  \fker(f, X, S)(u) \right]
  \end{tikzcd}
\]
where $\MS_{\calV}(X) = \{ (f, S) \in \Mor(\cate{Set}) \times \Ob(\calV): f \in \Hom_{\cate{Set}}(X, S) \}$. Then, 
\[
  \Theta(u) = \Theta(v) \Leftrightarrow c_{\calV}(X)(u) = c_{\calV}(X)(v),
\]
and it follows that 
\[
  \bmrm{F}(X) /  c_{\calV}(X) \simeq \im \Theta \subset \prod_{(f, S) \in \MS_{\calV}(X)} \bmrm{F}(X) / \fker(f, X, S) \in \Ob \calV,
\]
which implies $\bmrm{F}(X) /  c_{\calV}(X) \in \Ob \calV$. Hence, $\begin{tikzcd}[column sep = small]
   X
      \arrow[r, hook]
    &\bmrm{F}(X)
      \arrow[r, two heads]
    &\bmrm{F}(X) / c_{\calV}(X)
\end{tikzcd}$ consists the $\calV$-free algebra of $X$.






\begin{theorem}
  As a set, $\calV$ is a variety if and on if it is an equational class $\ec_{\calV}(X, R)$. 
\end{theorem}
\begin{proof}
  Given a set $X$ and a relation $R \subset \bmrm{F}(X)^2$, we assert that $\ec_{\calV}(X, R)$ is a variety. The proof proceeds in three parts:
  \begin{enumerate}
    \item for any $ S\in \ec_{\calV}(X, R)$ and subalgebra  $T \subset S$, $T \in \ec_{\calV}(X, R)$, since for any $f \to T$ the following diagram commutes, and so $\forall (u,v) \in R (\bar f (u) = \bar f (v))$.
    \[
      \begin{tikzcd}
        X
          \arrow[r, hook]
          \arrow[rd, "f"'] 
        &\bmrm{F}(X)
          \arrow[d, dashed, "\bar{f}"]
          \arrow[rd, dashed, "\bar{f}"]
        &
        \\
        &T
          \arrow[r, hook, "\subset"]
        &S
      \end{tikzcd}
    \]
    \item For any $ S\in \ec_{\calV}(X, R)$ and the epimorphism  $\pi: S \to T$, $T \in \ec_{\calV}(X, R)$. For any $f: S \to T$, let $g \in \prod_{x \in X}\pi^{-1}f(x)$, then $X \xrightarrow{g} S \xrightarrow{\pi} T = f$. By the commutativity of the diagram below, $\bar f(u) = \bar f(v)$ for any $(u, v) \in R$.
    \[
      \begin{tikzcd}
        X
          \arrow[r, hook]
          \arrow[rd, "f"']
          \arrow[rdd, "\exists g"'] 
        &\bmrm{F}(X)
          \arrow[d, dashed, "\bar{f}"]
          \arrow[dd, dashed, bend left=50, "\bar g"]
        \\
        &T
        \\
        &S
          \arrow[u, two heads, "\pi"']
      \end{tikzcd}
    \] 
    \item Suppose $I$ is small and $S_i \in \ec_{\calV}(X, R)$, then so does $\prod_{i \in I}S_i$. This is from the following commutative diagram, in which $p_i$ is the projection, which implies that 
    \[
      \forall j \in I (p_j\bar f(u) = p_j \bar f(v)) \Leftrightarrow \bar f(u) = \bar f (v)
    \]
    for every $(u, v) \in R$.
    \[
      \begin{tikzcd}
        X
          \arrow[r, hook]
          \arrow[rd, "f"']
          \arrow[rdd] 
        &\bmrm{F}(X)
          \arrow[d, dashed, "\bar{f}"]
          \arrow[dd, dashed, bend left=50, "p_j \bar f"]
        \\
        &\prod_{i\in I}S_i
          \arrow[d, "p_j"]
        \\
        &S_i
      \end{tikzcd}
    \]
  \end{enumerate}

  Conversely, given a variety $\calV$ and a countably finite set $X$, we assert that $\calV = \ec_{\calV}(X, c_{\calV}(X))$. The containment $\calV \subset \ec_{\calV}(X, c_{\calV}(X))$ is easy to verify. If $(u, v) \in \rmc_{\calV}(X)$, then for any $S \in \Ob \calV$ and any morphism $f: X \to S$, $\bar f (u) = \bar f (v)$. This implies that every $S$ in $\calV$ is in $\ec_{\calV}(X, c_{\calV}(X))$. For the reverse, let $S \in \ec_{\calV}(X, c_{\calV}(X))$. Choose an infinite set $Y$ that $|Y| \geq |X|$ and $|Y| \geq |S|$. We shall show that $S$ is the image of an epimorphism $[\bmrm{F}_{\calV}(Y) \twoheadrightarrow S]$, and so $S \in \calV$. 

  Suppose $(u, v) \in c_{\calV}(Y)$, then $u,v \in \bmrm{F}(Y)$, each of them can be viewed as a series of finite length that every coordinate belongs to set $Y \sqcup \Omega \sqcup \{(, )\}$ respectively, and so could it be viewed as a function which has a finite domain contained in $\Z_{\geq 1}$ and a finite image contained in $Y \sqcup \Omega \sqcup \{(, )\}$. Let $Y_0 = (\im u \cup \im v) \cap Y$, clearly it is a finite set. Let $X_0 \subset X$, and we may find a function $\xi_0: X \hookrightarrow Y$ that satisfies $\xi_0(X_0) = Y_0$.

  Since $X \xrightarrow{\xi_0} Y $ is an injection, there exists a left inverse of $\xi_0$ such that $X \xrightarrow{\xi_0} Y \xrightarrow{\eta_0} X = \idd_X$. The image of these two mappings under the functor $\bmrm{F}$ are denoted as $\xi$ and $\eta$ respectively. Thus, under the category of set, we have the following commutative diagram.
  \[
    \begin{tikzcd}
      X_0
        \arrow[r, hook]
        \arrow[d, "\xi_0|_{X_0}"']
      &X
        \arrow[r, hook]
        \arrow[d, "\xi_0"]
        \arrow[dd, bend right=50, "\idd"']
      &U\bmrm{F}(X)
        \arrow[d, "U\xi"]
        \arrow[dd, bend left=50, "\idd"]
      \\
      Y_0
        \arrow[r, hook]
      &Y
        \arrow[r, hook]
        \arrow[d, "\eta_0"]
      &U\bmrm{F}(Y)
        \arrow[d, "U\eta"]
      \\
      &X
        \arrow[r, hook]
      &U\bmrm{F}(X)
    \end{tikzcd}
  \]
\end{proof}
By the construction, there exists $u_0, v_0$ in $\bmrm{F}(X)$ such that $\xi(u_0) = u$ and $\xi(v_0) = v$, and it follows that $\eta(u) = u_0, \eta(v) = v_0$. 



From the diagram below and the universal property of $\bmrm{F}(Y)$, $\overline{c_{\calV}(X)\eta\iota} =  c_{\calV}(X)\eta$. Since $(u, v) \in c_{\calV}(Y)$, namely, $\bar{f}(u) = \bar{f}(v)$ for any $f: Y \to S$, we obtain that $c_{\calV}(X)\eta(u) = c_{\calV}(X)\eta(v)$. Hence $(u_0, v_0) \in c_{\calV}(X)$.
\[
  \begin{tikzcd}
    Y
      \arrow[r, hook, "\iota"]
      \arrow[rdd]
    &U\bmrm{F}(Y)
      \arrow[d, "\eta"]
      \arrow[dd, bend left=50, dashed, "\exists !"]
    \\
    &U\bmrm{F}(X)
      \arrow[d, "c_{\calV}(X)"]
    \\
    &U\bmrm{F}_{\calV}(X)
  \end{tikzcd}
\]


For any epimorphism $\psi: \bmrm{F}(Y) \twoheadrightarrow S$, again, by the universal property of $\bmrm{F}(X)$, we obtain that $\overline{\psi\xi\iota} = \psi\xi$. since $S \in \ec_{\calV}(X, c_{\calV}(X))$ and $(u_0, v_0) \in c_{\calV}(X)$, we may conclude that $\psi\xi(u_0) = \psi\xi(v_0)$, thus, $\psi(u) = \psi(v)$.
\[
  \begin{tikzcd}
    X
      \arrow[r, hook, "\iota"]
      \arrow[rdd]
    &U\bmrm{F}(X)
      \arrow[d, "\xi"]
      \arrow[dd, bend left=50, dashed, "\exists !"]
    \\
    &U\bmrm{F}(Y)
      \arrow[d, two heads, "\psi"]
    \\
    &US
  \end{tikzcd}
\]

Now we've established that for any $Y$ that not smaller than $X$ and $S$, and for any epimorphism $\bmrm{F}(Y) \twoheadrightarrow S$, $c_{\calV}(Y) \subset \ker \psi$. Hence, there exist a morphism that makes the following diagram commutes, and so $S \in \calV$.
\[
  \begin{tikzcd}
    \bmrm{F}(Y)
      \arrow[r, two heads, "\psi"]
      \arrow[d, "c_{\calV}(Y)"']
    &S
    \\
    \bmrm{F}_{\calV}(Y)
      \arrow[ur, dashed, "\exists !"']
    &
  \end{tikzcd}
\]


